\documentclass[12pt,letterpaper]{article}
\usepackage{geometry}
\usepackage[utf8]{inputenc}
\usepackage{setspace} 
\geometry{left=2.50cm, right=2.50cm, top=2.50cm, bottom=2.50cm}
\bibliographystyle{plain} 
\usepackage{fancyhdr}
\pagestyle{fancy} 
\pagenumbering{arabic}
\bibliographystyle{plain} 
\usepackage[spanish]{babel}
\begin{document}
	\spacing{1.5} 
	\lhead{}
	\chead{}
	\rhead{Caracas, 27/09/2018}
	\lfoot{}
	\cfoot{}
	\rfoot{Equipo de Desarrollo Tecnológico}
	
	
	\noindent \Large \textbf{Fundación Bit\&Nibs}\\
	\noindent \textbf{Equipo de Desarrollo Tecnológico}\\
	
	\normalsize
	
	 \noindent \textbf{Análisis de posicionamiento (SEO) para Blockchain.}\\
	 
	 \noindent La tecnología Blockchain esta creciendo de manera muy rápida, es por eso que  Bit\&Nibs como Fundación que adopta los Smart Contracts para llevar la trazabilidad del Cacao, debe conocer que palabras son las más acertadas para realizar publicaciones y posicionar sus post en la web y en las  Redes Sociales.\\
	
	\noindent Mediante el buscador de Instagram se obtuvo una data de cuantas publicaciones se realizan empleando los hashtag de las palabras que se nombran a continuación. Al día miércoles 28/09/2018 entre 10:00 pm y 11:00pm (Hora de Venezuela). Se obtuvieron los siguientes resultados:  \\
	
	\begin{itemize}
		\item Blockchain (1.7m publicaciones)
		\item Bitcoin    (4.3m publicaciones)
		\item Cryptocurrency (2.3m publicaciones)
		\item Tecnología  (1.7m publicaciones)
		\item Minar (24.5k publicaciones)
		\item Token (271k publicaciones)
		\item Minero (21k publicaciones)
		\item Smart contracts (35.3k publicaciones)
		\item Ethereum (1.2m publicaciones)
		\item Finanzas (442k publicaciones)
		\item Internet (2.7m publicaciones)
		\item Peer to peer (19k publicaciones)
		\item P2P(117k publicaciones)
		\item Startups (2.4m publicaciones) 
		\item Wallet (5.4m publicaciones) \textbf{Nota:} Se muestran publicaciones de billeteras físicas.
		\item ICO (548k publicaciones)
	\end{itemize}

\noindent Estas palabras fueron extraídas de los documentos \cite{deloitte} \cite{asobancaria} \cite{equisoft} \cite{bbva}, de dicho documentos se seleccionaron un compendio de palabras que aunque en Instragram no se posicionan bien al nivel de hashtag, son de uso común en el mundo de la Blockchain y las criptomonedas:

\begin{itemize}
	\item Moneda virtual.
	\item Pagos.
	\item Transacción.
	\item Tecnología disruptiva.
	\item Nodos.
	\item Red descentralizada.
	\item Hash.
	\item Proof of work.
	\item Encriptación.
	\item Autenticación.
	\item Confianza.
	\item Activos digitales.
	\item Código.
	\item Protocolos.
	\item Inmutable.
\end{itemize}

\noindent Estas palabras responde a descripciones del funcionamiento y aplicación de la Blockchain y criptomonedas, como se mencionó anteriormente, son palabras que están en los escritos del tema.\\

\noindent \textbf{Temas de sugerencia para estudio}\\

\noindent Entre los documentos consultados se denotó que existen ciertos temas que podrían ser abarcados en post y contenido respecto a Blokchain, tales como:

\begin{itemize}
	\item La Fundación Ethereum.
	\item El desconocimiento de la Blockchain.
	\item El Internet de las cosas IoT.
	\item Proyecto Hyperledger
	\item Proyecto R3
\end{itemize}

\begin{thebibliography}{}
	
	\bibitem{deloitte} Piscini E., Hyman G., Henry W. (2017) \textit{Blokchain: Economía de confianza}, Deloitte University Press, versión español Colombia. Recuperado el 27 de septiembre de 2018:  https://miethereum.com/blockchain/pdf
	
	\bibitem{asobancaria} Montoya G. (2017) \textit{Blokchain: Mirando más allá del Bitcoin}, AsoBancaria, Colombia. Recuperado el 27 de septiembre de 2018: https://miethereum.com/blockchain/pdf
	
	\bibitem{equisoft} Georges J. (2017) \textit{Blokchain: Una tecnología disruptiva con el poder de revolucionar el mundo financiero}, Equisoft, Estados Unidos. Recuperado el 27 de septiembre de 2018: https://miethereum.com/blockchain/pdf
	
	\bibitem{bbva} BBVA (2016) \textit{Tecnología Blockchain.}, BBVA, España. Recuperado el 27 de septiembre de 2018: bhttps://miethereum.com/blockchain/pdf
	
	
\end{thebibliography}

\end{document}
